%! TEX root = 0-main.tex
\chapter{Symmetry and Relativity}
\section{Symmetry and Conservation}
Recall the lagrangian can be written
\[L(\dot q, q, t) = \frac{1}{2}\dot q^2 - V(q)\]
The Lagrangian exhibits a symmetry if it is invariant under a transformation. Consider a translation \(q = q+\delta q\). Then, for the Lagrangian to exhibit symmetry, we must have
\[0 = \delta L = \pder{L}{\dot q}\delta \dot q + \pder{L}{q}\delta q\]
This comes from the total differential of \(L\). Substituting Euler-Lagrange into the second term,
\[0 = \pder{L}{\dot q}\delta\dot q + \der{}{t}\left(\pder{L}{\dot q}\right)\delta q = \der{}{t}\left[\pder{L}{\dot q}\delta q\right]\]
Thus, the quantity
\[Q = \pder{L}{\dot q}\delta q\]
is conserved.
This is \emph{Noether's Theorem}---for every symmetry in the lagrangian, there is a corresponding conserved quantity.

\subsection{Gallilean Group}
Newton's laws are invariant under the Gallilean group---the group generated by time translation, 3 spatial translations, 3 rotations, and 3 Gallilean boosts. Recall a gallilean boost is given
\[x'^i = x^i+v^i t\qquad t' = t\]
and that the line element is given
\[\d{s}^2 =\delta_{ij}\d{x^i}\d{x^j}\]
Trivially, we can indeed show that Newton's law is invariant under such a transformation.


However, we immediately see one new relation: velocities transform as
\[u'^i = u^i+v^i\]
thus, the speed of light does not transform as we observe; it is not invariant under a gallilean boost. Similarly, Maxwell's equations are not invariant under such a transformation.

\section{Special Relativity}
Using the invariance of the speed of light, we can derive the new invariant line element using Einstein's Clock as well as a new symmetry group for the laws of physics.

Consider two mirrors separated by a distance \(L\). We bounce light back and forth, perpendicular to the two surfaces. The time it takes light for a round trip is trivially \(\Delta t = 2L/c\). However, when an obsever moves parallel to the surface of the mirrors at a velocity \(v\), as an external observer, they no longer see light travelling perpendicularly; it rather zigzags between the two. The new round-trip length the light must take is now \(L'=2\sqrt{L^2+(v\Delta t'/2)^2}\). Thus, we can show easily that
\[-(c\Delta t)^2 + (\Delta x')^2 = -(c\Delta t)^2+(\Delta x)^2\]
More generally, we find the invariant line element becomes
\begin{equation}
	\d{s}^2 \equiv -(c\d{t})^2+(\d{x})^2+(\d{y})^2+(\d{z})^2
\end{equation}
We will often use natural units in which \(c=1\). Note, we are using the metric signature \((-+++)\). We define the \emph{Minkowski metric} for flat space-time as
\begin{equation}
	\d{s}^2 = \eta_{\mu\nu}\d{x^\mu}\d{x^\nu}
\end{equation}
where
\begin{equation}
	\eta_{\mu\nu}=\begin{bmatrix}
		-1 & 0 & 0 & 0\\
		 0 & 1 & 0 & 0 \\
		 0 & 0 & 1 & 0\\
		 0 & 0 & 0 & 1
	\end{bmatrix}
\end{equation}

\subsection{Lorentz Boost}
Consider a boost
\[x = vt+x'\]
Because \(x'\) is some constant, we find that we must have some propotionality constant such that
\[x' = \gamma(x-vt)\]
By symmetry, we also find that because \(x' = x-vt\) we have
\[x = \gamma(x'+vt')\]
Using these two expressions, we find that
\[t' = \gamma\left[t+\frac{1-\gamma^2}{v\gamma^2}x\right]\]
and enforcing the invariant interval, we find that
\begin{equation}\gamma = \frac{1}{\sqrt{1-\beta^2}}\qquad\qquad \beta = \frac{v}{c}\end{equation}
and
\begin{subequations}
	\begin{align}
		ct'&=\gamma(ct-\beta x)\\
		x'&=\gamma(x-vt)
	\end{align}
\end{subequations}
Setting \(c=1\), we ifnd that we can rewrite this as
\begin{equation}
	\begin{bmatrix}
		t'\\
		x'
	\end{bmatrix} = \begin{bmatrix}
	\gamma & -\gamma v\\
	-\gamma v & \gamma
	\end{bmatrix} \begin{bmatrix}
		t\\x
	\end{bmatrix}
\end{equation}
Because the line element is preserved, the determinant of the matrix is one, and indeed, we have
\[\gamma^2-v^2\gamma^2 = \frac{1}{1-v^2} - \frac{v^2}{1-v^2}=1\]
We define \(\gamma=  \cosh\psi\) for some \emph{rapidity} \(\psi\), which shows us that \(\sinh\psi = -v\gamma\), using the hyperbolic trig identity
\[\cosh^2\psi -\sinh^2\psi = 1\]
Thus, we obtain a representation of the lorentz boost in terms of a hyperbolic rotation:
\begin{equation}
	\Lambda_\nu^\mu = \begin{bmatrix}
		\cosh\psi & \sinh\p\\\sinh\p&\cosh\p
	\end{bmatrix}
\end{equation}
or, in Einstein notation,
\begin{equation}
	x'^\mu = \Lambda_\nu^\mu x^\nu
\end{equation}
Inserting this into the inner product, we find
\[
	\eta = \Lambda\T\eta\Lambda
\]
or
\begin{equation}
	\eta_{\rho\sigma} = \Lambda_\rho^\mu\Lambda_\sigma^\nu\eta'_{\mu\nu}
\end{equation}
just like what we had for rotations earlier.

\subsection{Light cones and Causality}
Recall our metric. A line element is time-like if \(\d{s}^2<0\), time-like if \(\d{s}^2>0\), and null if \(\d{s}s^2 = 0\). Finite intervals can more simply be determined by examining the light cone, given by 
\[(ct)^2 = x^2+y^2+z^2\]
Time-like separation is contained within the light cone and space-like separation is outside of the light cone. Objects can only interact with objects within their past or future light cones; if two objects have a space-like separation, they can never interact, and there is no causal connection. In this view, Gallilean space-time is the limit where \(c\to\infty\).

We define the proper time to be defined by the invariant line element, as we can arbitrarily fix the space-dimensions \(\d{x^i} = 0\).
\begin{equation}
	\d\tau^2 = -\d{s}^2
\end{equation}
This value is then the same in all frames. Thus, \(\tau\) gives a natural parametrization for a worldline. We can write
\[\tau =\int_A^B\d\tau = \int\sqrt{\d{t}^2-\d{\vb x}^2} = \int\d\tau \sqrt{\left(\der{t}{\tau}\right)^2 - \left(\der{\vb x}{\tau}\right)^2} = \int\d{t}\sqrt{1-\left(\der{\vb x}{t}\right)^2}\]
It is important to note that in flat space-time \emph{a straight-line path maximises proper time}. Further, for light, \(\d{\tau}^2 = 0\).
