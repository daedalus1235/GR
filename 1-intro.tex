%! TEX root = 0-main.tex
\chapter{Introduction}
\section{Why GR?}
Consider the following conundrum:
\[m_I\der{\vb x}{t2}=\vb F\]
we see that the inertial mass term coincides with the mass term in Newton's law of gravitation:
\[F = \frac{G M}{r^2}m_g\]
This is known as the Weak Equivalence Principle.
There is no apparent reason why these two masses need be related so directly. Yet, numerous experiments show that these two masses to be identical to at least one part in \SI{e13}.

Another unintuitive observation is that contrary to the Newtonian view, an object at rest on a table experiences an acceleration while a falling object experiences no acceleration.
In an inertial frame, in which the object \emph{experiences} no acceleration (such as an object in a parabolic trajectory), we see that gravity vanishes; gravity is not a true force but a fictitious force. Rather than gravity, the acceleration of an object that we observe is dictated by the curvature of space-time, and the requirement to travel along a ``straight-line path.'' Even on earth, we can see that a laser pointer is deflected by a gravitation field, \emph{despite having no mass}. We will see that mass defines the curvature of space-time, while the curvature determines how the masses will move.
